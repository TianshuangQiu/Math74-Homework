\documentclass[12pt]{article}
\usepackage[usenames]{color} %used for font color
\usepackage{amsmath, amssymb, amsthm}
\usepackage{wasysym}
\usepackage[utf8]{inputenc} %useful to type directly diacritic characters
\usepackage{graphicx}
\usepackage{caption}
\usepackage{subcaption}
\usepackage [english]{babel}
\usepackage [autostyle, english = american]{csquotes}
\MakeOuterQuote{"}
\graphicspath{ {./} }
\newcommand{\Z}{\mathbb{Z}}
\newcommand{\N}{\mathbb{N}}
\newcommand{\R}{\mathbb{R}}
\newcommand{\Q}{\mathbb{Q}}
\newcommand{\prob}{\mathbb{P}}
\newcommand{\degrees}{^{\circ}}


\author{Tianshuang (Ethan) Qiu}
\begin{document}
\title{Math 74, Week 6}
\maketitle

\section{Mon Lec, 1a}
Base case: $n=1, n=2, a_1=3, a_2=5$, base cases hold.
\newline
Assume that for some $n \in \N$, all $m \in \N 1, 1 \leq m \leq n$ satisfies this identity, we have
$$a_{n+1}=3a_n-2a_{n-1}=3(2^n+1)-2(2^{n-1}+1)=3\times 2^n+3-2\times 2^{n-1}-2$$
$$=2^{n-1}(6-2)+1=2^{n-1}(2^2)+1=2^{n+1}+1$$
Thus we have proven the inductive case. Q.E.D.

\section{Mon Lec, 3c}
Base case: $n=0, m=0, f_n=1, f_m=1, f_{n+m+1}=2$, base case holds.
\newline
For any $n, m \in \N$, assume that the statement is true for all $n'\leq n, m' \leq m$. Using our direct formula we have
$$f_{n+1}=\frac{1}{\sqrt 5} (\phi^{n+1} - \bar \phi^{n+1})$$
$m+1$ has a similar logic. Now consider $n+1$ and $f_{n+m+2}$.
\newline
Using the direct formula, we know that $f_{n+m+2}=1/\sqrt5 (\phi^{n+m+2} - \bar \phi^{n+m+2})$
$$\frac{f_{n+m+2}}{f_{n+m+1}} = \frac{\phi^{n+m+2} - \bar \phi^{n+m+2}}{\phi^{n+m+1} - \bar \phi^{n+m+1}}$$
\newline
Now we take $(f_{n+1}f_m+f_{n+2}f_{m+1})/(f_{n}f_{m}+f_{n+1}f_{m+1})$
$$=\frac{(\phi^{n+1}-\bar\phi^{n+1})(\phi^{m}-\bar\phi^{m})+(\phi^{n+1}-\bar\phi^{n+2})(\phi^{n+1}-\bar\phi^{m+1})}{(\phi^{n}-\bar\phi^{n})(\phi^{m}-\bar\phi^{m})+(\phi^{n+1}-\bar\phi^{n+1})(\phi^{m+1}-\bar\phi^{m+1})}$$
We can expand and cancel the terms through long division, finally leaving $$\frac{\phi^{n+m+2} - \bar \phi^{n+m+2}}{\phi^{n+m+1} - \bar \phi^{n+m+1}}$$, which is equal to the above fraction. Since $f_{n+m+1}=f_nf_m+f_{n+1}f_{m+1}$ by inductive hypothesis, we have now proven the inductive step. Q.E.D.

\section{Mon Lec, 4a}

\section{Mon Dis, 6}
First we claim that $n^2-(n+1)^2-(n+2)^2+(n+3)^2=4$
\end{document}
