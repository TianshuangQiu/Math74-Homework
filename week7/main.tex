\documentclass[12pt]{article}
\usepackage[usenames]{color} %used for font color
\usepackage{amsmath, amssymb, amsthm}
\usepackage{wasysym}
\usepackage[utf8]{inputenc} %useful to type directly diacritic characters
\usepackage{graphicx}
\usepackage{caption}
\usepackage{subcaption}
\usepackage{float}
\usepackage{mathtools}
\usepackage [english]{babel}
\usepackage [autostyle, english = american]{csquotes}
\MakeOuterQuote{"}
\graphicspath{ {./} }
\newcommand{\Z}{\mathbb{Z}}
\newcommand{\N}{\mathbb{N}}
\newcommand{\R}{\mathbb{R}}
\newcommand{\Q}{\mathbb{Q}}
\newcommand{\prob}{\mathbb{P}}
\newcommand{\degrees}{^{\circ}}
\DeclarePairedDelimiter\ceil{\lceil}{\rceil}
\DeclarePairedDelimiter\floor{\lfloor}{\rfloor}

\author{Tianshuang (Ethan) Qiu}
\begin{document}
\title{Math 74, Week 6}
\maketitle

\section{Mon Lec, 4c}
Non of the number repeating means that it is a rearrangement of the set of remainders $\{1,2,...6\}$. We first show that if $a \not \equiv b \bmod 7, \text{then} 4a \not \equiv 4b \bmod 7$
\newline
Proof:
Suppose for contradiction that $4a \equiv 4b \bmod 7$, then we have $7 \mid 4(a-b)$. Since $\gcd(4,7)=1$, $7 \mid a-b$, $a \equiv b \bmod 7$, \lightning. We have a contradiction, therefore $4a \not \equiv 4b \bmod 7$
\newline
Let $a=1, b = 2,3,4,5,6$. We can see that $(4 \times 1)...(4 \times 6) \equiv 6! \bmod 7$. Since 6! is coprime with 7, we can divide it, leaving us with
$$4^6 \equiv 1 \bmod 7$$

\section{Mon Lec, 5a}
Using the formula
$$\phi(n) = n \prod _{p \mid n}(1-\frac{1}{p})$$
$$\phi(10) = 10(1-\frac{1}{2})(1-\frac{1}{5}) = 4$$

\section{Mon Lec, 6}
We simplify
$$\frac{x+2k}{3} \leq x+1$$
$$x+2k \leq 3x+3$$
$$2x \geq 2k-3$$
$$x \geq \frac{2k-3}{2}$$
So $(2k-3)/2 = 3, 2k = 9, k = 4.5$
\newpage


\section{Mon Dis, 2b}
$$17 ^ {1707} \bmod 11 \equiv 6^{1707}\bmod 11$$
By fermats little theorem $6^10 \equiv 1 \bmod 11$
$$6^1707 \equiv (6^10)^170 \times 6^7 \bmod 11 \equiv 6^7 \bmod 11$$
Now we start raising 6 to higher powers.
$$6^2 \equiv 3 \bmod 11$$
$$6^4 \equiv 9 \bmod 11$$
$$6^7 \equiv 3\times9\times6 \bmod 11 \equiv 8 \bmod 11$$
So $17 ^ {1707} \equiv 8 \bmod 11$

\section{Mon Dis, 4c}



\end{document}
