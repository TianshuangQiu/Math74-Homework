\documentclass[12pt]{article}
\usepackage[usenames]{color} %used for font color
\usepackage{amsmath, amssymb, amsthm}
\usepackage{wasysym}
\usepackage[utf8]{inputenc} %useful to type directly diacritic characters
\usepackage{graphicx}
\usepackage{caption}
\usepackage{subcaption}
\usepackage{float}
\usepackage{mathtools}
\usepackage [english]{babel}
\usepackage [autostyle, english = american]{csquotes}
\MakeOuterQuote{"}
\graphicspath{ {./} }
\newcommand{\Z}{\mathbb{Z}}
\newcommand{\N}{\mathbb{N}}
\newcommand{\R}{\mathbb{R}}
\newcommand{\Q}{\mathbb{Q}}
\newcommand{\prob}{\mathbb{P}}
\newcommand{\degrees}{^{\circ}}
\DeclarePairedDelimiter\ceil{\lceil}{\rceil}
\DeclarePairedDelimiter\floor{\lfloor}{\rfloor}

\author{Tianshuang (Ethan) Qiu}
\begin{document}
\title{Math 74, Week 7}
\maketitle

\section{Mon Lec, 4c}
Non of the number repeating means that it is a rearrangement of the set of remainders $\{1,2,...6\}$. We first show that if $a \not \equiv b \bmod 7, \text{then} 4a \not \equiv 4b \bmod 7$
\newline
Proof:
Suppose for contradiction that $4a \equiv 4b \bmod 7$, then we have $7 \mid 4(a-b)$. Since $\gcd(4,7)=1$, $7 \mid a-b$, $a \equiv b \bmod 7$, \lightning. We have a contradiction, therefore $4a \not \equiv 4b \bmod 7$
\newline
Let $a=1, b = 2,3,4,5,6$. We can see that $(4 \times 1)...(4 \times 6) \equiv 6! \bmod 7$. Since 6! is coprime with 7, we can divide it, leaving us with
$$4^6 \equiv 1 \bmod 7$$

\section{Mon Lec, 5a}
Using the formula
$$\phi(n) = n \prod _{p \mid n}(1-\frac{1}{p})$$
$$\phi(10) = 10(1-\frac{1}{2})(1-\frac{1}{5}) = 4$$

\section{Mon Lec, 6}
We simplify
$$\frac{x+2k}{3} \leq x+1$$
$$x+2k \leq 3x+3$$
$$2x \geq 2k-3$$
$$x \geq \frac{2k-3}{2}$$
So $(2k-3)/2 = 3, 2k = 9, k = 4.5$
\newpage


\section{Mon Dis, 2b}
$$17 ^ {1707} \bmod 11 \equiv 6^{1707}\bmod 11$$
By fermats little theorem $6^{10} \equiv 1 \bmod 11$
$$6^{1707} \equiv (6^{10})^{170} \times 6^7 \bmod 11 \equiv 6^7 \bmod 11$$
Now we start raising 6 to higher powers.
$$6^2 \equiv 3 \bmod 11$$
$$6^4 \equiv 9 \bmod 11$$
$$6^7 \equiv 3\times9\times6 \bmod 11 \equiv 8 \bmod 11$$
So $17 ^ {1707} \equiv 8 \bmod 11$

\section{Mon Dis, 4c}
We first compute $\phi(100)$
$$\phi(100)=100(1-\frac{1}{2})(1-\frac{1}{5}) = 40$$
Since $\gcd(43,100)=1$, we can apply Euler Totient theorem.
$$43^{1763} \equiv (43^{40}){44} \times 43^3 \equiv 43^3 \bmod 100$$
$43^2 = 1849 \equiv 49 \bmod 100$, so $43^3 \equiv 49 \times 43 \bmod 100$
Finally we get $43^3 \equiv 7 \bmod 100$
\newpage


\section{Wed Lec, 2a}
We factor the expression into $n(n^2-1)=n(n+1)(n-1)$
\newline
Integers that are divisible by 2 are spaced such that there is one every other, and those divisible by 3 are spaced such that there is one every third. Here we have a product of 3 consecutive integers, so there must be at least 1 integer between $n-1, n, n+1$ that is divisible by 2, and at least 1 that is divisible by 3. Since $6=2 \times 3$, we have $6 \mid n(n+1)(n+2)$.
\newline
Thus we have proven that $6 \mid n^3-n$
\newline
$\blacksquare$

\section{Wed Lec, $2021^{4042} \bmod 100$}
First we since 2021 is odd and does not end in 5 or 0, $\gcd(2021,100)=1$, so we can apply the Euler Totient Theorem.
$$\phi(100)=100(1-\frac{1}{2})(1-\frac{1}{5}) = 40$$
$$2021^{4042} \equiv (2021^{40})101 \times 2021^2 \equiv 2021^2 \equiv 21^2 \bmod 100$$
Now we can simply calculate the square to be $441 \equiv 41 \bmod 100$
\newpage


\section{Wed Dis 1a}
First we show that $17n^2+1 \not \equiv 0 \bmod 4$
\newline
$17n^2+1 \equiv n^2+1 \bmod 4$ Now we assume that there exists some $n \geq 1$ that sataisfies this equation. Then we have $n^2+1 \equiv 0 \bmod 4$
$$n^2 \equiv 3 \bmod 4$$
Now we consider all the possibilities of $n \mod 4$: $1^2 \equiv 1, 2^2 \equiv 0, 3^3 \equiv 1, 4^2 \equiv 0 (\bmod 5)$. Therefore there is no way $n$ that sataisfies $n^2\equiv 3$. Our assumption is incorrect and $4 \not \mid 17n^2+1$
\newline
We repeat the proof for $\bmod 5$
\newline
$17n^2+1 \equiv 2n^2+1 \bmod 5$ We assume that there exists some $n \geq 1$ that sataisfies this equation. Then we have $2n^2+1 \equiv 0 \bmod 5$
$$2n^2 \equiv 4 \bmod 5$$
$$n^2 \equiv 2 \bmod 5$$
We can examine the possible squares: $1^2 \equiv 1, 2^2 \equiv 4, 3^2 \equiv 4, 4^2 \equiv 1 (\bmod 5)$ Once again there is no possible way for $n^2$ to be 2. Our assumption is incorrect and $5 \not \mid 17n^2+1$
\newline
Thus we have proven both parts of the hypothesis.
$\blacksquare$

\section{Wed Dis, 5}
$n \equiv$ the sum of its digits modulo 9 because when in decimal, we can express $n = \sum 10^kd$, where d is the digit at the kth place. We know that $10^k \equiv 1(\bmod 9)$, so each digit is equivalent to being multiplied by 1, which is just itself.
\newline
For this problem we take the modulo of $9$, which as shown above is the sum of all the digits. We can then find which one is missing based on the fact that the sum is $45$.
\newline
By Fermat's Little Theorem $2^8 \equiv 1 \bmod 9$
$$2^29 \equiv (2^{8})^3 \times 2^5 \equiv 2^5 \bmod 9$$
We can then compute the rest by brute force: $2^5 = 32 \equiv 5 \bmod 9$
\newline
Since the sum is $45 \equiv 0 \bmod 9$, having a remainder of $5$ means that we are missing its complement: $9-5=4$.
\newline
Therefore $2^29$ is missing the number $4$.
\newpage


\section{Fri Lec, 2c}
Fir we define a process where an even amount of $a$ 2's match up to form $a/2$ 0's. The rest match up with others to form $k-a$ 1's. Regardless of what $a$ is, the oddity of the amount of 1's is preserved. Since $(0,0)\to 0, (1,1)\to 0, (2,2)\to 0, (0,1) \to 1, (0,2)\to 1, (1,2)\to 1$, we can see that if $b$ amount of 2's match up to form $b/2$ 0's, we would have $k-b$ 1's.
\newline
Since $a,b$ are always even, the remaining amount will be odd if the original amount of 2's: $k$ is odd, and even otherwise. Let the amount of 1's be $m$, $m+(k-a) \bmod 2$ will always be the same regardless of what moves are taken.
\newline
Now we have reduced the problem to one that we have completed in class, where the game only has 1's and 0's. Now, the oddity of the 1's is preserved, and we can use that to come to a conclusion: if $k+m$ is even, there would be an even number of $1$'s, so the final number will be $0$; otherwise there would be an odd number of $1$'s, and the final number will be $1$.

\section{Fri Lec, 3c}
Let the amount of red chameleon's be $r$, greens be $g$, yellows be $y$. Whenever two of different color meet, they change to the third color. Take the differences of $r,y$, if the two chameleons are of these colors then they both lose 1, preserving the difference. Otherwise one of them is decreased by 1 while the other is increased by 2, so the difference increases by 3, which is equivalent modulo 3.
\newline
We can repeat this argument for all other color pairs. Therefore $r-y, g-r, y-g$ remains constant $\bmod 3$.
\newline
Now consider this problem. If it is possible to chanage every chameleon into a single color, aand since $2015+1013+572 \equiv 0 \bmod 3$ then $g-y \equiv 0$ or is . Now we check the differences $g-y = 2015-1013 = 1002 \equiv 0 \bmod 3$. $g-r = 2015-572=1443 \equiv 0$. $y-r = 1013-572=441\equiv 0$. Since all the differences are in line with what we stated, it is possible.
\newline
We match $147$ green chameleons with $147$ yellows, this creates $294$ reds, now we have $1868$ greens, $866$ yellows, $866$ reds. Now we simply match each red with each yellow, turning them all into greens. At the end we have $3600$ green chameleons, and the process is complete.


\section{Fri Lec, 4}
We know that the invariant of this machine is the order of the numbers: if $a<b$ are inputed as a pair, then the output $a',b'$ will still have $a'<b'$
\newline
Secondly, the largest odd factor of the difference is preserved. In this question, $(7,29)$ has $11$ as the largest odd factor.

\subsection{a}
$$(7,29) \to (8,30) \to (9, 31) ... \to (7+22, 29+22) = (29, 51)$$
Thererfore it is possible.

\subsection{b}
$$(7,29) \to (8,30) \to (4, 15) \to (4+86, 15+86) = (90, 101)$$
Thererfore it is possible.

\subsection{c}
$$(7,29) \to (7+30,29+30) = (37, 59) \to (37+33,59+33) = (70, 92)$$
$$(7,29) \to (8,30) \to (4, 15) \to (4+55, 15+55) = (59,70)$$
$$((37, 59), ((59,70)) \to (37, 70)$$
$$((37, 70), ((70, 92)) = (37, 92)$$
Thererfore it is possible.

\subsection{d}
$7 < 29$ but $39 > 27$, the order is flipped, so it is not possible.

\subsection{e}
$$(7,29) \to (7+33,29+33) = (40, 62) \to (40+22,62+22) = (62, 84)$$
$$((40, 62), (62, 84)) \to (40, 84)$$
Thererfore it is possible.

\end{document}
