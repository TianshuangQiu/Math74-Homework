\documentclass[12pt]{article}
\usepackage[usenames]{color} %used for font color
\usepackage{amsmath, amssymb, amsthm}
\usepackage{wasysym}
\usepackage[utf8]{inputenc} %useful to type directly diacritic characters
\usepackage{graphicx}
\usepackage{caption}
\usepackage{subcaption}
\usepackage{float}
\usepackage{mathtools}
\usepackage [english]{babel}
\usepackage [autostyle, english = american]{csquotes}
\MakeOuterQuote{"}
\graphicspath{ {./} }
\newcommand{\Z}{\mathbb{Z}}
\newcommand{\N}{\mathbb{N}}
\newcommand{\R}{\mathbb{R}}
\newcommand{\Q}{\mathbb{Q}}
\newcommand{\prob}{\mathbb{P}}
\newcommand{\degrees}{^{\circ}}
\DeclarePairedDelimiter\ceil{\lceil}{\rceil}
\DeclarePairedDelimiter\floor{\lfloor}{\rfloor}

\author{Tianshuang (Ethan) Qiu}
\begin{document}
\title{Math 74, Week 14}
\maketitle


\section{Mon Lec, 3a}
Let $a$ be the length of this rectangle that is opposite the wall, and $b$ be the length of the other side. So we have $a+2b \leq 36$, and we try to maximize $ab$. We can simplify the first equation to $a = 36 - 2b$
\newline
By AM-GM, we have that $\frac{2a+b}{2} \geq \sqrt{2ab}$, now since we know that $a+2b \leq 36$, we can subsitute that in.
$$18 \geq \frac{2a+b}{2} \geq \sqrt{2ab}$$
Thus the maximum the area $ab$ can be is $18^2/2 = 162$. Now we try to find an $a,b$ such that $ab = 162$.
Let $a = 9, b = 18$, and $ab=162$, achieving the maximum.

\section{Mon Lec, 5a}
We manipulate $2 \sqrt x > 3 - \frac{1}{x}$, and it is equivalent to showing
$$2 \sqrt{x} + \frac{1}{x} \geq 3$$
Then apply AM-GM to see that $\frac{\sqrt{x}+\sqrt{x}+1/x}{3} \geq
\sqrt[3]{\sqrt{x}\sqrt{x}1/x} = 1$. Rearranging this gives
$$2 \sqrt{x} + \frac{1}{x} \geq 3$$
Thus we have proven the statement.


\section{Mon Lec, 6}
We say two inequalities are equivalent when they are true and false at the same time.
\newline
In the first equation $(x-a)^2+1>0$, $(x-a)^2$ is non-negative, so the statement is always true.
$$4ax^2 + 4x + 1 > 0$$
$$(4a-4)x^2 + (2x+1)^2 > 0$$
This inequality holds true when $(4a-4)>0$, so the two inequalities are equivalent when $a > 1$
\newpage


\section{Mon Dis, 1a}
By AM-GM, $\frac{a+b}{2} \geq \sqrt{ab}$, $\frac{b+c}{2} \geq \sqrt{bc}$, etc.
\newline
We can multiply these equations to get
$$\frac{(a+b)(b+c)...(e+a)}{2^5} \geq \sqrt{a^2b^2c^2d^2e^2}$$
$$(a+b)(b+c)(c+d)(d+e)(e+a) \geq 32abcde$$


\section{Mon Dis, 1b}
By AM-CM, $(\sum_{n=1}^{2021} n) /2021 \geq \sqrt[2021]{2021!}$. We can apply the arithmetic series sum to the lhs:
$$\frac{(2021+1)2021}{2 \times 2021} \geq \sqrt[2021]{2021!}$$
$$(\frac{2022}{2}) ^ {2021} \geq 2021!$$
Thus it is proven.


\section{Mon Dis, 3c}
Let our box have dimensions $h,w,l$ for height, width, and length. Now we know by AM-GM that $\frac{h+w+l}{3} \geq \sqrt[3]{whl}$. Since this box is inscribed on the ellipsoid, the edge in the octant 1 with coordinates $(w/2,l/2,h/2)$ satisfies $$\frac{w^2}{4a^2}+\frac{l^2}{4b^2}+\frac{h^2}{4c^2} = 1$$
Now we manipulate the equation: $w^2b^2c^2+l^2a^2c^2+h^2a^2b^2 = 4a^2b^2c^2$

\end{document}
