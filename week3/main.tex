\documentclass[12pt]{article}
\usepackage[usenames]{color} %used for font color
\usepackage{amssymb} %maths
\usepackage{amsmath} %maths
\usepackage[utf8]{inputenc} %useful to type directly diacritic characters
\usepackage{graphicx}
\usepackage [english]{babel}
\usepackage [autostyle, english = american]{csquotes}
\MakeOuterQuote{"}
\graphicspath{ {./} }
\newcommand{\Z}{\mathbb{Z}}
\newcommand{\N}{\mathbb{N}}
\newcommand{\R}{\mathbb{R}}
\newcommand{\Q}{\mathbb{Q}}
\newcommand{\prob}{\mathbb{P}}

\author{Tianshuang (Ethan) Qiu}
\begin{document}
\title{Math 74, Week 3}
\maketitle

\section{Wed Lec, 1a}
Prove that $\binom{n}{k} + \binom{n}{k+1} = \binom{n+1}{k+1}$
\subsection{algebraic}
We first expand the expression:
$$LHS = \frac{n!}{(n-k)!k!} + \frac{n!}{(n-k-1)!(k+1)!}$$
Then we simplify:
$$LHS = \frac{n!}{(n-k)(n-k-1)!k!} + \frac{n!}{(k+1)(n-k-1)!k!}$$
$$LHS = \frac{(n!(k+1))+(n!(n-k))}{(n-k)(k+1)(n-k-1)!k!} $$
$$LHS = \frac{n!(n+1)}{(k+1)!(n-k)!} $$
$$LHS = \frac{(n+1)!}{(k+1)!(n-k)!}$$
Now we expand the right side:
$$RHS = \frac{(n+1)!}{(k+1)!(n-k)!}$$
We see thatt RHS = LHS.
\newline
Q.E.D.

\subsection{Combinatorial}
The right hand side calculates the number of bitstrings of length $n+1$ with $k+1$ 0's. Since each bit can either end in $0$ or $1$, we can split them into different cases.
\newline
If the the last is $0$, there are $k$ 0's left in the substring of length $n$. So to calculate this, we use $\binom{n}{k}$.
\newline
If the last bit is $1$, there are still $k+1$ 0's left in the substring before it. Using the formula, we get $\binom {n}{k+1}$
\newline
Adding them together equals the right hand side. They are evaluating the same thing.
\newline
Q.E.D.

\section{Wed Lec, 4b}
Let people be the pigeons and each letter is a different hole.
\newline
Number of pigeons = number of names = 33
\newline
Number of holes = number of letters - 30
\newline
Since the amount of pigeons is greater than the amount of holes, some hole must have at least 2 pigeons by PHP. Therefore some ending letter must be shared by at least 2 names.

\section{Wed Lec, 5}

\subsection{a}
We can form a segment from any two points. So the total amount of segments is $\binom{50}{2} = \frac{50!}{2!48!}=1225$

\subsection{b}
Since no three points are colinear, we can choose any 3 points to form a triangle. So the total amount is $\binom{50}{3} = \frac{50!}{3!47!} = 19600$

\subsection{c}
Similar to part (b), we choose any 4 points to form a quadrilateral. The total amount is $\binom {50} {4} = \frac{50!}{46!4!} = 230300$.

\newpage

\section{Wed Dis, 2}
Since the three books needs to be next to each other in the specified order, we can think of it as if the three are "glued together" into one book. So essentially we are looking for the amount of ways to arrange 5 books.
\newline
There are 5 ways to choose the first one, 4 for the second, ..., for a total of $5! = 120$ ways to arrange it.

\section{Wed Dis, 4}
We are looking for the amount of ways to rearrange "GAUSS". Since there are 5 letters, we have $5!=120$. Then we remove the ways that we overcount the repeat "S". $\frac{5!}{2!} = 60$
\newline
Similarly, for "RAMANUJAN", we take all possible permutations and divide by repeating letts "A" and "N"$: \frac{9!}{3!2!}$

\newpage




\end{document}
