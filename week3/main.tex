\documentclass[12pt]{article}
\usepackage[usenames]{color} %used for font color
\usepackage{amssymb} %maths
\usepackage{amsmath} %maths
\usepackage[utf8]{inputenc} %useful to type directly diacritic characters
\usepackage{graphicx}
\graphicspath{ {./} }
\newcommand{\Z}{\mathbb{Z}}
\newcommand{\N}{\mathbb{N}}
\newcommand{\R}{\mathbb{R}}
\newcommand{\Q}{\mathbb{Q}}
\newcommand{\prob}{\mathbb{P}}

\author{Tianshuang (Ethan) Qiu}
\begin{document}
\title{Math 104}
\maketitle

\section{Wed Lec, 1a}
Prove that $\binom{n}{k} + \binom{n}{k+1} = \binom{n+1}{k+1}$
\subsection{algebraic}
We first expand the expression:
$$LHS = \frac{n!}{(n-k)!k!} + \frac{n!}{(n-k-1)!(k+1)!}$$
Then we simplify:
$$LHS = \frac{n!}{(n-k)(n-k-1)!k!} + \frac{n!}{(k+1)(n-k-1)!k!}$$
$$LHS = \frac{(n!(k+1))+(n!(n-k))}{(n-k)(k+1)(n-k-1)!k!} $$
$$LHS = \frac{n!(n+1)}{(k+1)!(n-k)!} $$
$$LHS = \frac{(n+1)!}{(k+1)!(n-k)!}$$
Now we expand the right side:
$$RHS = \frac{(n+1)!}{(k+1)!(n-k)!}$$
We see thatt RHS = LHS.
\newline
Q.E.D.

\subsection{Combinatorial}
The right hand side calculates the number of bitstrings of length $n+1$ with $k+1$ 0's. Since each bit can either end in $0$ or $1$, we can split them into different cases.
\newline
If the the last is $0$, there are $k$ 0's left in the substring of length $n$. So to calculate this, we use $\binom{n}{k}$.
\newline
If the last bit is $1$, there are still $k+1$ 0's left in the substring before it. Using the formula, we get $\binom {n}{k+1}$
\newline
Adding them together equals the right hand side. They are evaluating the same thing.
\newline
Q.E.D.

\end{document}
