\documentclass[12pt]{article}
\usepackage[usenames]{color} %used for font color
\usepackage{amsmath, amssymb, amsthm}
\usepackage{wasysym}
\usepackage[utf8]{inputenc} %useful to type directly diacritic characters
\usepackage{graphicx}
\usepackage{caption}
\usepackage{subcaption}
\usepackage{float}
\usepackage{mathtools}
\usepackage [english]{babel}
\usepackage [autostyle, english = american]{csquotes}
\MakeOuterQuote{"}
\graphicspath{ {./} }
\newcommand{\Z}{\mathbb{Z}}
\newcommand{\N}{\mathbb{N}}
\newcommand{\R}{\mathbb{R}}
\newcommand{\Q}{\mathbb{Q}}
\newcommand{\prob}{\mathbb{P}}
\newcommand{\M}{\mathcal{M}}
\newcommand{\degrees}{^{\circ}}
\DeclarePairedDelimiter\ceil{\lceil}{\rceil}
\DeclarePairedDelimiter\floor{\lfloor}{\rfloor}

\author{Tianshuang (Ethan) Qiu}
\begin{document}
\title{Math 74, Week 10}
\maketitle

\section{Mon Lec, 6c}
$$z^{n-1}-1 = \prod_{k=0}^{n-1}(z-\omega_k)$$
As shown in class, the complex roots are evenly spaced across the unit circle, $2\pi/n$ apart. So we have
$$\omega_k = e^{2\pi i \frac{k}{n}}$$


\section{Mon Lec, 6f}
$$(z-1)(z^{n-1}+z^{n-2}+...+z^{2}+z+1) = (z^n+z^{n-1}+...+z^2+z)-(z^{n-1}+z^{n-2}+...+z+1)=z^n-1$$
Therefore we can switch our statement to $\frac{z^n-1}{z-1}$
\newline
Now we subsitute our answer from the previous question in, and since $\omega_0=1$, it cancels out with the first term.
\newline
We can factor the expression into
$$\prod_{k=1}^{n-1}(z-\omega_k)$$
where
$$\omega_k = e^{2\pi i \frac{k}{n}}$$
Essentially the same as 6c but with $z=1$ removed.


\section{Mon Lec, 7b}

\subsection{6c}
Sum: $-\frac{0}{1}=0$
\newline
Product: $(-1)^n\frac{1}{1} = (-1)^n$

\subsection{6f}
Sum: $-\frac{1}{1}=-1$
\newline
Product: $(-1)^n\frac{0}{1} = 0$
\newpage


\section{Mon Dis, 1a}
$$|A_0A1|...|A_0A_8|=\prod_{k=0}^{8}|1-\omega_k|=|\prod_{k=0}^{8}(1-\omega_k)|$$
The last equivalency is due the fact that multiplication of the modulus is equal to the modulus of the product.
\newline
We have proven above that $(z-1)(z^{n-1}+z^{n-2}+...+z^{2}+z+1)=z^n-1$, so consider
$$\frac{z^n-1}{z-1}=\frac{(z-1)(z^{n-1}+z^{n-2}+...+z^{2}+z+1)}{z-1}=z^{n-1}+z^{n-2}+...+z^{2}+z+1$$
Now using the roots of the polynomial we know that $z^9-1=(z-1)(z-\omega)...(z-\omega^8)$, in this case we have divided out $z-1$, so we have
$$z^8+z^7+...+1=(z-\omega)...(z-\omega^8)$$
Let $z=1$, and we have $9=(z-\omega)...(z-\omega^8)$, since $|9|=9$, we have shown that $|A_0A1|...|A_0A_8|=9$


\section{Mon Dis, 3f}
Let $x=y-2$, so $x^3=y^3-6y^2+12y-8$. Now we plug $y$ back
$$y^3-6y^2+12y-8 = -6(y-2)^2-12y+24-6$$
$$y^3-6y^2+12y-8 = -6y^2+12y-6$$
$$y^3 = 2$$
Now since we know that $2^3=8$, we can directly solve:
$y_1= \sqrt[3]{2}, x_1=\sqrt[3]{2}+2$
\newline
Then we factor $(y^3)/(y-\sqrt[3]{2})=y^2+\sqrt[3]{2}y+\sqrt[3]{4}$ Now we apply the quadratic formula to get
$$y_2=\frac{-1-\sqrt3i}{\sqrt[3]{2}}, y_3=\frac{-1+\sqrt3i}{\sqrt[3]{2}}$$
So we have $x_2=\frac{-1-\sqrt3i}{\sqrt[3]{2}}-2, x_3=\frac{-1+\sqrt3i}{\sqrt[3]{2}}-2$

\end{document}
