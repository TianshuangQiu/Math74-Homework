\documentclass[12pt]{article}
\usepackage[usenames]{color} %used for font color
\usepackage{amsmath, amssymb, amsthm}
\usepackage{wasysym}
\usepackage[utf8]{inputenc} %useful to type directly diacritic characters
\usepackage{graphicx}
\usepackage{caption}
\usepackage{subcaption}
\usepackage{float}
\usepackage{mathtools}
\usepackage [english]{babel}
\usepackage [autostyle, english = american]{csquotes}
\MakeOuterQuote{"}
\graphicspath{ {./} }
\newcommand{\Z}{\mathbb{Z}}
\newcommand{\N}{\mathbb{N}}
\newcommand{\R}{\mathbb{R}}
\newcommand{\Q}{\mathbb{Q}}
\newcommand{\prob}{\mathbb{P}}
\newcommand{\M}{\mathcal{M}}
\newcommand{\degrees}{^{\circ}}
\DeclarePairedDelimiter\ceil{\lceil}{\rceil}
\DeclarePairedDelimiter\floor{\lfloor}{\rfloor}

\author{Tianshuang (Ethan) Qiu}
\begin{document}
\title{Math 74, Week 10}
\maketitle

\section{Mon Lec, 6c}
$$z^{n-1}-1 = \prod_{n=0}^{n-1}(z-\omega_n)$$
As shown in class, the complex roots are evenly spaced across the unit circle, $2\pi/n$ apart. So we have
$$\omega_k = e^{2\pi i \frac{m}{n}}$$


\section{Mon Lec, 6f}
$$(z-1)(z^{n-1}+z^{n-2}+...+z^{2}+z+1) = (z^n+z^{n-1}+...+z^2+z)-(z^{n-1}+z^{n-2}+...+z+1)=z^n-1$$
Therefore we can switch our statement to $\frac{z^n-1}{z-1}$
\newline
Now we subsitute our answer from the previous question in, and since $\omega_0=1$, it cancels out with the first term.
\newline
We can factor the expression into
$$\prod_{n=1}^{n-1}(z-\omega_n)$$
where
$$\omega_k = e^{2\pi i \frac{m}{n}}$$
Essentially the same as 6c but with $z=1$ removed.


\section{Mon Lec, 7b}


\end{document}
