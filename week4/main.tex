\documentclass[12pt]{article}
\usepackage[usenames]{color} %used for font color
\usepackage{amssymb} %maths
\usepackage{amsmath} %maths
\usepackage[utf8]{inputenc} %useful to type directly diacritic characters
\usepackage{graphicx}
\usepackage [english]{babel}
\usepackage [autostyle, english = american]{csquotes}
\MakeOuterQuote{"}
\graphicspath{ {./} }
\newcommand{\Z}{\mathbb{Z}}
\newcommand{\N}{\mathbb{N}}
\newcommand{\R}{\mathbb{R}}
\newcommand{\Q}{\mathbb{Q}}
\newcommand{\prob}{\mathbb{P}}

\author{Tianshuang (Ethan) Qiu}
\begin{document}
\title{Math 74, Week 4}
\maketitle

\section{Lec Mon, 1c}
\subsection{a}
Since each term is the product of $x^a, y^b, z^c$, and $a+b+c = 2020$, we can simplify this problem into dogs and biscuits. $\binom{2020+3-1}{3-1} = \binom{2022}{2}$

\subsection{b}
Before combining, we expand each term by picking one variable from each of the 2020 $(x+y+z)$ multiplied together. So we have $3^2020$.

\subsection{c}
We can reach the same result by subtracting the amount where there is only $x$, or only $y$, or only $z$, or $xy$, $xz$, $yz$.
\newline
For the first three, there is only 1 way for that to happen since that variable has to be raised to 2020. For $xz$, we have $a+b = 2020$, feeding 2018 biscuits to 2 dogs. We need to subtract 2 since we have already counted having only one term. Therefore $\binom{2018+2-1}{2-1} = 2019$.
\newline
Adding them together we have $1 \times 3 + 2019 \times 3 = 6060$
\newpage


\section{Dis Mon, 1a}
LHS is the amount of ways to choose a team with $k$ people and a captain from a group with $n$ people. It chooses the team first: $\binom{n}{k}$. Then from that team we choose a captain with $k$ ways to do it.
\newline
RHS calculates the amount of ways to choose a captain first: $n$, then the rest of the team: $\binom{n-1}{k-1}$. Both sides calculate the same thing. Therefore LHS = RHS.
\newline
Q.E.D.

\section{Dis Mon, 4}
$$x+\frac{1}{x}=7$$
$$(x+\frac{1}{x})^2=49$$
$$x^2+\frac{1}{x^2}+2=49$$
$$x^2+\frac{1}{x^2}=47$$







\end{document}
