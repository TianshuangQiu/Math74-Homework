\documentclass[12pt]{article}
\usepackage[usenames]{color} %used for font color
\usepackage{amsmath, amssymb, amsthm}
\usepackage{wasysym}
\usepackage[utf8]{inputenc} %useful to type directly diacritic characters
\usepackage{graphicx}
\usepackage{caption}
\usepackage{subcaption}
\usepackage{float}
\usepackage{mathtools}
\usepackage [english]{babel}
\usepackage [autostyle, english = american]{csquotes}
\MakeOuterQuote{"}
\graphicspath{ {./} }
\newcommand{\Z}{\mathbb{Z}}
\newcommand{\N}{\mathbb{N}}
\newcommand{\R}{\mathbb{R}}
\newcommand{\Q}{\mathbb{Q}}
\newcommand{\prob}{\mathbb{P}}
\newcommand{\degrees}{^{\circ}}
\DeclarePairedDelimiter\ceil{\lceil}{\rceil}
\DeclarePairedDelimiter\floor{\lfloor}{\rfloor}

\author{Tianshuang (Ethan) Qiu}
\begin{document}
\title{Math 74, Week 8}
\maketitle
\section{Mon Lec, 5}


\section{Mon Lec, 6}



\section{Mon Dis, 2c}
Let $a\leq b \leq c \leq d$, we can do this since $\N$ is ordered. We rearrange $a+b+c+d = abcd$ to $a+b+c = d(abc-1)$
\newline
Since we have assumed an order, we have $a+b+c \leq 3d$, so $abc-1 \leq 3$, $abc \leq 4$
\newline
Now in this case $abc$ can only be $1,2,3,4$. We consider each of these cases:
\newline
$abc = 1 \implies a=1, b=1, c=1$, solving for our original equation gives $3 = 0d$, which does not have a solution.
\newline
$abc = 2$, by our assumed order we have $a=1, b=1, c=2$, and we solve $2d=4+d$, $d=4$, this gives a solution.
\newline
$abc = 3$, by our assumed order we have $a=1, b=1, c=3$, and we solve $3d=5+d$, which does not have an integer solution.
\newline
$abc = 4$, by our assumed order we have $a_0=1, b_0=1, c_0=4$, or $a_1=1, b_1 = 2, b_2=2$. Solving the first equation gives $d = 2$, which contradicts our assumption of $c \leq d$, so this is not a solution. Solving the latter gives $4d = 5+d$, which does not have an integer solution.
\newline
So finally we have our solution: rearrangements of $1,1,2,4$ is the only natural solution to this system.

\section{Mon Dis, 4g}
$$(7,29) \to (7+1,29+1) = (8, 30) \to (4, 15)$$
$$(4,15) \to (4+11, 15+11) = (15, 26)$$
$$(7,29) \to (7+19, 29+19) = (26, 48)$$
$$((4, 15), (15,26)) \to (4, 26)$$
$$((4,26), (26,48)) \to (4,48)$$
$$(4, 48) \to (2, 24) \to (1, 12)$$

\section{Wed Dis, 3a}

\section{Wed Dis, 3b}

\section{Wed Dis, 4a}


\section{Fri Lec, 4a}

\section{Fri Lec, 4b}

\section{Fri Lec, 4c}

\section{Fri Lec, 4d}

\end{document}
