\documentclass[12pt]{article}
\usepackage[usenames]{color} %used for font color
\usepackage{amsmath, amssymb, amsthm}
\usepackage{wasysym}
\usepackage[utf8]{inputenc} %useful to type directly diacritic characters
\usepackage{graphicx}
\usepackage{caption}
\usepackage{subcaption}
\usepackage{float}
\usepackage{mathtools}
\usepackage [english]{babel}
\usepackage [autostyle, english = american]{csquotes}
\MakeOuterQuote{"}
\graphicspath{ {./} }
\newcommand{\Z}{\mathbb{Z}}
\newcommand{\N}{\mathbb{N}}
\newcommand{\R}{\mathbb{R}}
\newcommand{\Q}{\mathbb{Q}}
\newcommand{\prob}{\mathbb{P}}
\newcommand{\degrees}{^{\circ}}
\DeclarePairedDelimiter\ceil{\lceil}{\rceil}
\DeclarePairedDelimiter\floor{\lfloor}{\rfloor}

\author{Tianshuang (Ethan) Qiu}
\begin{document}
\title{Math 74, Week 12}
\maketitle

\section{Mon Lec, 5}
\subsection{a}
$|K|=6$, let $e$ be the identity transformation and $r$ be a rotation of $\frac{360}{6}=60 \degrees$. The other elements are: $r^2, r^3, r^4, r^5$

\subsection{b}
Since $D_6$ is the group of all symmetries, $K$ is only the rotational ones, and $D_6$ also contains reflective symmetries. Therefore $K$ is a subgroup of $D_6$

\subsection{c}
Let our generator $\omega = r$, $\omega ^ 0 = e, \omega^1 = r, \omega^2 = r^2$, and so on. It is a generator because its powers can generate all the elements of our gorup.

\subsection{d}
Let our generator $\omega = r^5$, $\omega^0 = e, \omega^1 = r^5, \omega^2 = r^4, \omega^3=r^3$ etc.


\end{document}
