\documentclass[12pt]{article}
\usepackage[usenames]{color} %used for font color
\usepackage{amsmath, amssymb, amsthm}
\usepackage{wasysym}
\usepackage[utf8]{inputenc} %useful to type directly diacritic characters
\usepackage{graphicx}
\usepackage [english]{babel}
\usepackage [autostyle, english = american]{csquotes}
\MakeOuterQuote{"}
\graphicspath{ {./} }
\newcommand{\Z}{\mathbb{Z}}
\newcommand{\N}{\mathbb{N}}
\newcommand{\R}{\mathbb{R}}
\newcommand{\Q}{\mathbb{Q}}
\newcommand{\prob}{\mathbb{P}}
\newcommand{\degrees}{^{\circ}}


\author{Tianshuang (Ethan) Qiu}
\begin{document}
\title{Math 74, Week 5}
\maketitle

\section{Mon Dis, 1b}

\subsection{}


\subsection{}

$$\prod^n_i=2 (1-\frac{1}{n^2})$$
We examine $1-1/k^2$ and factor it into $\frac{k^2-1}{k^2} = \frac{(k+1)(k-1)}{k^2}$. Since k is incrementing by 1 in our series, we can cancle each term out. We can expand our series into
$$\frac{1\times 3}{2^2} \frac{2 \times 4}{3^3} ... \frac{(n-1)(n+1)}{n^2}$$
$$= \frac{1}{2} \frac{n+1}{n} = \frac{n+1}{2n}$$

\end{document}
